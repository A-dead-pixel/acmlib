\chapter{Utils}

\myimport{headers}
\myrawimport[-l sh]{headers/bazshrc.sh}
\myrawimport[-l raw]{headers/vimrc}
\myrawimport[-l sh]{headers/sprawdzaczka.sh}
% \myimport{example-code}

\chapter{Podejścia}

\begin{itemize}[noitemsep]
	\item dynamik, zachłan
	\item sposób "liczba dobrych obiektów = liczba wszystkich obiektów - liczba złych obiektow"
	\item czy warunek konieczny = warunek wystarczający?
	\item odpowiednie przekształcenie równania
	\item zastanowić się nad łatwiejszym problemem, bez jakiegoś elementu z treści
	\item sprowadzić problem do innego, łatwiejszego/mniejszego problemu
	\item sprowadzić problem 2D do problemu 1D (szczególny przypadek: zamiatanie; częsty przypadek: niezależność wyniku dla współrzędnych X od współrzędnych Y)
	\item konstrukcja grafu
	\item określenie struktury grafu
	\item optymalizacja bruta do wzorcówki
	\item czy można poprawić (może zachłannie) rozwiązanie nieoptymalne?
	\item czy są ciekawe fakty w rozwiązaniach optymalnych? (może się do tego przydać brute)
	\item sprawdzić czy w zadaniu czegoś jest "mało" (np. czy wynik jest mały, albo jakaś zmienna, może się do tego przydać brute)
	\item odpowiednio "wzbogacić" jakiś algorytm
	\item cokolwiek poniżej $10^9$ operacji ma szansę wejść
	\item co można wykonać offline? Coś można posortować? Coś można shuffle'ować?
	\item narysować dużo swoich własnych przykładów i coś z nich wywnioskować
\end{itemize}
