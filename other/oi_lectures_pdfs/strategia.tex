\documentclass{article}
\usepackage[utf8]{inputenc}
\usepackage[T1]{fontenc}
\usepackage[a4paper, total={460pt, 700pt}]{geometry}
\usepackage{array}
\usepackage[polish]{babel}
\renewcommand{\arraystretch}{1.25}
 
\usepackage{tabularx}
\usepackage{fancyhdr}
\usepackage[lastpage,user]{zref}
\usepackage{enumitem}
\usepackage{graphicx}
\usepackage{float}
\usepackage{calc}
\usepackage{scalerel}
\usepackage{fancyvrb}
\usepackage{longtable}
\usepackage{booktabs}
\usepackage{hyperref}
\hypersetup{colorlinks=true, urlcolor=blue}
 
\let\oldverbatim\verbatim
\let\oldendverbatim\endverbatim
\def\verbatim{\Verbatim[samepage=true]}
\def\endverbatim{\endVerbatim}
 
\let\OldIncludegraphics\includegraphics
\renewcommand{\includegraphics}[2][]{\OldIncludegraphics[scale=0.55, #1]{#2}}
 
\providecommand{\tightlist}{%
    \setlength{\itemsep}{0pt}\setlength{\parskip}{0pt}}
 
\setcounter{secnumdepth}{0}

\begin{document}

\section*{Wykład o strategii (Tomasz Nowak)}

\begin{itemize}
    \item Przypomnienie zasad drugiego etapu OI
    \begin{itemize}
        \item pytania tylko w pierwszych dwóch godzinach, odpowiedzi tak / nie / niepoprawne pytanie / odpowiedź wynika z treści zadania / bez odpowiedzi. Pytania techniczne przez całe zawody
        \item często wchodzić na pytania (o tym jak Michał nie zauważył zmiany podzadania na BOI)
        \item drukowanie i kopia zapasowa, skrypt submit (o tym, jak chyba dwa lata temu przesunięto o godzinę drugi etap, bo w jednym mieście padło zasilanie kompów z live cd)
        \item max 15 submitów, liczy się ostatni submit
        \item w plikach są przykładowe i limity czasu
        \item dokumentacja
    \end{itemize}
    
    \item Minimum, by się dostać do finału
    \begin{itemize}
        \item ostatnie progi: 200, 162, 200, 87, 80, 168, 140, 206, 175, 62. Łatwo wyciągnąć wnioski
        \item w pierwszym dniu oczywiście, najlepiej celować w wzorcówkę i bruta albo dwa bruty, nie da się niczego mądrego powiedzieć
        \item zależnie od wyniku w pierwszym dniu, strategia się zmienia w drugim dniu (sytuacja wygląda tak, że powinno się sklepać wzorcówkę, albo można na spokojnie bezpieczniej dwa bruty)
        \item bardzo kluczowe jest rozpoznanie łatwiejszego zadania, które zazwyczaj jest (Justynka na drugim etapie OIG (nie OI) wyzerowała bo klepała 2h hardkora)
        \item nieprzewidywalność i niepewność i brak wiedzy
    \end{itemize}
    
    \item Idealny ,,early game`` podczas contestu
    \begin{itemize}
        \item *zrozumieć* *oba* zadania podczas pierwsze pół godziny, albo nawet godziny. Nigdy nie kminić tylko nad jednym zadaniem. Nigdy nie dopuścić do sytuacji, gdzie się klepie zadanko, a drugie jest jeszcze nieprzeczytane (jakieś przykłady takich błędów z życia)
        \item zrobić wiele swoich przykładów, zamiast przykładowych z treści (podchwytliwe treści i przykładowe)
        \item potwierdzić, że się rozumie treść, na podstawie przykładowych
        \item nigdy nie zaszkodzi przeczytać treść jeszcze raz, od początku do końca (pierwszy dzień warsztatów staszicowych)
        \item zrozumieć każdy szczegół treści, szczególnie dziwne warunki (po co one są albo co ułatwiają)
    \end{itemize}
    
    \item Doświadczenie w rozwiązywaniu zadań
    \begin{itemize}
        \item słabą intuicję można nadrobić dużym doświadczeniem
        \item rozwiązanie może korzystać z techniki lub obserwacji, którą znamy
        \item intuicja - czujemy z jakiego podejścia/przekształcenia należy skorzystać
        \item doświadczenie - mamy dużą listę pomysłów, z których można skorzystać w danym zadaniu
        \item dużo osób ma spore doświadczenie, ale nie zdaje sobie z tego sprawy (jak je wykorzystać?)
        \item może warto przed OIem przejrzeć przez zadania, które się wbiło, i przypomnieć sobie rozwiązania i wyciągnąć z nich całą ,,esencję''? Dla każdego zadania skrócić jak najbardziej ideę rozwiązania, w taki sposób, że gdyby się sobie powiedziało o tym, kiedy się kminiło zadanko, to to by znacząco pomogło
        \item ostatnio zadania są częściej na techniki, niż obserwacje. Oznacza to większy nacisk na doświadczenie
        \item przykład: znajomość problemów NP-trudnych
    \end{itemize}
    
    \item Lista technik to złoto
    \begin{itemize}
        \item przedstawić kilka zadanek na każde podejście
        \item dlaczego lista technik może czasem znacząco pomóc w rozwiązaniu zadania
        \item podkreślić, że lista technik to złoto
        \item nakrzyczeć, by znali na pamięć podejścia, gdyż może to pomóc
        \item wyjaśnić, że nie ma powodu się jej wstydzić
    \end{itemize}
    
    \item Pewność siebie - dlaczego czasem się (niepotrzebnie) poddajemy
    \begin{itemize}
        \item Szczecin to zadupie, nie mamy osób, z którymi możemy się porównywać
        \item jesteśmy lepsi, niż nam się wydaje
        \item moja historyjka na finale OI - wbiłem jedno zadanko, pomyślałem ,,mm, ładne i trudne zadanie, po to tu przyjechałem. Ok, jestem już zadowolony ze swojego wyniku''. Co z tego wyszło? Nie wbiłem łatwych, a na pewno mnie na to było stać. Patrzyłem na te zadania i wgl nie miałem podejścia by je wbić na 100.
        \item często się zdarza, że zatrzymujemy się przy łatwiejszym zadanku i się poddajemy, a jesteśmy w stanie zrobić nawet trudniejsze. ,,Flow''
        \item pewność siebie może mieć wpływ na wynik na conteście
        \item jak budować pewność siebie? Porobić / przypomnieć sobie zadanka drugoetapowe
        \item nie daj się przerazić treści, często autorzy specjalnie próbują przerazić
        \item nie bój się algorytmu, tylko go sklep
        \item celować wyżej
        \item załóż, że zadania są łatwe
    \end{itemize}
    
    \item Co zrobić, gdy ma się laga mózgu?
    \begin{itemize}
        \item opcja 1: nie walczyć z tym, tylko przeczekać. Zrobić sobie przerwę, pójść zjeść kanapkę, pójść do toalety
        \item opcja 2: spróbować ,,zresetować'' kminę, tzn spojrzeć na zadanie od nowa, przypomnieć sobie jakie obserwacje zostały zrobione (albo jakie obserwacje by się przydały), przypomnieć sobie w głowie listę technik (szczególnie listę podejść), spróbować z nich maksymalnie skorzystać, spróbować przypomnieć sobie podobne zadanka, które się robiło, albo podobne podejścia. Powołać się na swoje doświadczenie i wgl
    \end{itemize}
    
    \item Dobry ,,midgame`` podczas contestu
    \begin{itemize}
        \item klepać najpierw bruty (nie zawsze ale zazwyczaj tak) - dlaczego to jest bardzo bardzo bardzo dobre (pomaga przy dużej liczbie kwestii i sytuacji), mamy już na starcie jakieś punkciki
        \item dobre jest klepać kolejne podzadania, jednak czasem podzadania to bait
        \item nie zapominać, że są dwa zadania, oraz nie wiemy które jest łatwiejsze
        \item często zdarza mi się, że nic nie mam podczas pierwszych trzech godzin contestu, a potem maxuję (narysować wykres ,,ile mi się wydaje że mam'' oraz ,,ile w praktyce wymyśliłem'')
        \item rozwiązywanie zadań często u mnie wygląda tak, że próbuję wyczerpać pomysły/podejścia/techniki które mi się przyjdą do głowy z doświadczenia, albo szukam obserwacji (przy czym wiem jakie mniej więcej szukać), zazwyczaj to się udaje
        \item nie stresować się xd (wiem że łatwo powiedzieć), naprawdę nie jest koniec świata jak się obudzimy w połowie i pomyśli się ,,ej, ale ja jeszcze nie mam, shit, no to koniec''
    \end{itemize}
    
    \item Jak nie marnować czasu?
    \begin{itemize}
        \item Justyna w pierwszej klasie na drugim etapie OI
        \item \textit{Jak macie geo to uciekajcie} - Kacper Walentynowicz
        \item W momencie, kiedy myślę, że powinienem klepać od nowa, jak tego nie zrobię to zawsze żałuję
        \item \textit{Czasami się opłaca klepać od nowa, zanim to będzie za późno} - Justyna Jaworska
        \item Klepać rozwiązania etapowo (kolejne podzadania), dać przykłady z pisania przedział-przedział. Sprawdzać kolejne części kodu. Jak się skończy fragment kodu, powinno się zastanowić czy na pewno jest git i czy można go przetestować, bo potem marnuje się czas na debuggowanie wszystkiego naraz (binsearch wielu rzeczy naraz po kodzie jest trudniejszy niż zwykły binsearch)
        \item spróbować przekminić kod, zanim się spróbuje go napisać (nie jest to dla każdego, ale spróbować nie zaszkodzi)
        \item wkopanie się może być z trzech powodów: zła implementacja albo zostawienie jednego zadania albo klepanie heury
        \item sztuczki implementacyjne: add i mul dla modulo, spróbować sprowadzić klepę do użycia znanego już rozwiązania, np. konstrukcja grafu i bfs
        \item nagłówki: rozsądna opcja podczas contestu. Można np napisać wypisywacz vectora intów, a jak potrzeba to rozszerzyć do template<class T>
        \item jak się jest w pewnym momencie niepewny czego się robi, to jest źle
        \item poddanie się boli, ale może się opłacić
        \item koksy kminią na zmianę nad zadaniami (przykład: tourist, ja, Michał)
    \end{itemize}
    
    \item Dobry ,,endgame'' podczas contestu
    \begin{itemize}
        \item \textit{Myśl dwa razy, abyś nie musiał kodzić trzy razy} - Karol Pokorski
        \item \textit{Jak widzę, że klepię gówno, to po prostu usuwam kod} - Marek Skiba
        \item jak *bardzo* ważne jest testowanie rozwiązań (Michał zerował część P w staszicowych contestach)
        \item można skorzystać z brutów sklepanych wcześniej, jednak bruty powinny nie mieć wspólnej części kodu z wzorcówką
        \item generować wszystkie możliwe testy, shuffle'ować też jeżeli to może mieć znaczenie 
        \item poza testowaniem, co można zrobić? Limity czasu oraz limity pamięci oraz overflow. Jak je sprawdzić?
        \item flagi kompilacji oraz narzędzie gdb (oraz oiejq OIa jeżeli jest)
        \item TODO rozszerzyć o klepaniu, konieczności testowania, błędach przy tym
        \item jak jest wiele testów w pliku, to najlepiej generować $t=2$
        \item na ich kompach, które są beznadziejne, najlepiej jest robić mało testów z dużym $t$ zamiast dużo z małym $t$ (nawet jak trzeba u siebie dodać istnienie $t$)
        \item często opłaca się bardziej zasubmitować bruta, niż nieprzetestowaną wzorcówkę (zależy od dużo rzeczy)
    \end{itemize}
    
    \item Co zapamiętać na pamięć przed OIem?
    \begin{itemize}
        \item listę technik, szczególnie podejścia
        \item flagi kompilacji
        \item sposób użycja gdb
        \item prostą wersję nagłówków (najlepiej swoją wersję)
        \item w jaki sposób sprawdzić czas i pamięć programu
        \item generatorka i sprawdzarka
    \end{itemize}
    
    \item Jak dużo wyciągnąć z tego obozu
    \begin{itemize}
        \item trzy OIowe zadanka bez odsłonięć, polecam pisać bruty. Czyli maksować EV wyniku zamiast nie patrzeć na prawdopodobieństwo zbugowania i celować w maxy
        \item spróbować wyciągnąć z zadań jak najwięcej esencji
        \item stały jest bardzo ważny, buduje praktykę i doświadczenie. Zazwyczaj nie pamięta się technik, których się nigdy nie klepało (zresztą to może powodować wkopienie się, tak jak Justynka w pierwszej klasie)
        \item warsztaty technikowe
        \item ponownie, stały jest bardzo ważny
    \end{itemize}
\end{itemize}

\end{document}

